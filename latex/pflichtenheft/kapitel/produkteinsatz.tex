
\section{Anwendungsgebiete}\label{sec:Anwendungsgebiete}
Das Produkt \projektname ist Teil einer Software, die eine einfache Möglichkeit bieten soll, zertifizierte Software für Bauprojekte auf einer offenen Plattform darzustellen.
Der \projektname soll hierbei eine Möglichkeit für den Benutzer bieten beliebig viele Dienste zu kombinieren.
Dann soll mittels einer Kompatibilitätsprüfung dem Nutzer mittgeteilt werden, ob die gewählte Kombination zulässig ist.


\section{Zielgruppen}\label{sec:Zielgruppen}

Die Nutzer des \projektname sind Personen, die hauptsächlich in der Baubranche (Architektur, Haustechnik etc.) arbeiten.
Dabei gibt es folgende Rollen: Administratoren, Benutzer und Benutzer ohne eigenen Account.

\subsection{Benutzer ohne Account}
Benutzer ohne Account sind in der Lage Dienste und öffentliche Kombinationen anzuschauen. Hierfür sind nur grundlegende Computerkenntnisse vonnöten, wie die Bedienung eines Internetbrowsers.

\subsection{Benutzer mit Account}
Benutzer mit Account haben die Möglichkeit Kombinationen zu erstellen, auf Kompatibilität zu prüfen, diese zu speichern und festzulegen, ob sie für alle Benutzer oder nur für bestimmte sichtbar und änderbar sind.
Diese sollten über grundlegende Computer-/Smartphonekenntnisse verfügen bzw. mit einem Browser/Smartphone umgehen können.

\subsection{Administratoren}
Die Administratoren haben zusätzlich zu den normalen Nutzerrechten noch die Möglichkeit neue Dienste manuell über die Webschnittstelle einzugeben und
Administratorenrechte an normale Nutzer zu vergeben. Die Administratoren sollten vertrauenswürdige Personen innerhalb der Organisation sein und mindestens über grundlegende Computer-/Smartphonekenntnisse verfügen.

\section{Betriebsbedingungen}\label{Betriebsbedingungen}

\subsection{Physikalische Umgebung}
Zum Betreiben des Produktes wird ein Server benötigt.
Als Nutzer des Produkts wird ein web-fähiger Computer bzw. ein web-fähiges Smartphone benötigt.

\subsection{Betriebszeit}
Das Produkt sollte durchgehend verfügbar sein.
Die Hauptlast wird vorraussichtlich zu den Kernarbeitszeiten von 8.00-17.00 Uhr sein.

\subsection{Datensicherung}
Die Daten werden jeden Montag um 04:00 Uhr gesichert.
