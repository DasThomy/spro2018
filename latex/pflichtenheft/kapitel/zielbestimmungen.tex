\section{Musskriterien}
\subsection{Webseite und App}
\begin{itemize}
\item Beide zeigen die Kompatibilität einer Kombination von Diensten an.
\item Man kann sich in den Account einloggen.
\item Man kann gespeicherte Kombinationen suchen, auswählen und ansehen.
\item Man kann Dienste über ihren Namen und ihre Tags suchen.
\end{itemize}

\subsection{Webseite}
\begin{itemize}
\item Administratoren können Dienste und Formate hinzufügen
\item Administratoren können Dienste bearbeiten und updaten, bei Formatänderungen muss ein neuer Dienst erstellt werden.
\item Administratoren können andere Nutzer zum Administrator ernennen.
\item Man kann einen Account mit Name, Vorname, Titel, Organisation, E-Mail Adresse und Passwort erstellen.
\item Eingeloggte Benutzer können Kombinationen erstellen, speichern und freigeben.
\item Die Freigabe für Kombinationen wird für einzelne Personen oder auf öffentlich gesetzt.
\end{itemize}

\subsection{App}
\begin{itemize}
\item Man kann Kombinationen per E-Mail versenden.
\end{itemize}

\section{Sollkriterien}
\subsection{Webseite}
\begin{itemize}
\item Man kann Gruppen erstellen und Kombinationen für diese freigeben.
\end{itemize}

\section{Kannkriterien}
\subsection{Webseite}
\begin{itemize}
\item Man kann sich Kopien der Kombinationen erstellen und diese dann verändern.
\item Man kann Kombinationen per E-Mail versenden.
\end{itemize}

\subsection{App}
\begin{itemize}
\item Administratoren können Dienste und Formate hinzufügen
\item Administratoren können Dienste bearbeiten und updaten, bei Formatänderungen muss ein neuer Dienst erstellt werden.
\item Administratoren können andere Nutzer zum Administrator ernennen.
\item Man kann einen Account mit Name, Vorname, Titel, Organisation, E-Mail Adresse und Passwort erstellen.
\item Eingeloggte Benutzer können Kombinationen erstellen, speichern und freigeben.
\end{itemize}

\section{Abgrenzungskriterien}
\begin{itemize}
\item Wir garantieren gute Performance bei Kombinationen mit bis zu 5 Diensten und bis zu 50 Nutzern gleichzeitig.
Darüber hinaus wird keine gute Performance gewährleistet.
\item Es wird keine Überprüfung stattfinden, ob die eigegebenen Daten der Dienste mit den offiziellen Spezifikationen übereinstimmen.
\item Wir stellen keine Schnittstellen bereit, um Kombinationen kompatibel zu machen.
\item Es wird nach dem Release keine Updates und keine Wartung geben.
\item Es wird nur eine Android-App geben und keine App für andere Betriebssysteme.
\item Bei Alternativvorschlägen von Kombinationen werden nicht mehr als 2 Dienste in eine Kette genommen.
\end{itemize}
